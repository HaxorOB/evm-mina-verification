\chapter{Appendix A. In-EVM Mina State}

This introduces a description for in-EVM Mina Protocol state handling mechanism
which is supposed to provide a bridge user with the way to verify plaintext 
transactions coming from Mina database commit log on EVM.

\section{Overview}

The protocol described literally replicates Mina's commit log constructon
protocol on EVM.

The overall process description is as follows:

\begin{algorithm}
    \caption{Commit Log Construction Overview}
    \label{commitlog}
    \begin{enumerate}
        \item A user retrieves a replication packet $B_{n}$ containing some trnasaction $T$ 
            from Mina's commit log.
        \item A user submits the replication packet $B_{n}$ to the in-EVM piece of logic.
        \item The in-EVM piece of logic emplaces the replication packet $B_{n}$
            into the backwards-linked list $C$.
        \item The in-EVM piece of logic computes a Poseidon hash $H_{B_{n}}$ of
            a replication packet $B_{n}$ and inserts such one in a Merkle Tree
            $T$.
        \item The in-EVM piece of logic uses a Merkle Tree's hash $H_{B_{n}}$ of
            a particular replication packet $B_{n}$ as an input to the
            state proof verification mechanism, taking the state proof from the
            original Mina's cluster in the same time, corresponding to the
            replication packet $B_{n}$ seqno.
        \item In case the verification of a state proof corresponding to the
            replication packet $B_{n}$ was completed sucessfully,
            such a replication packet $B_{n}$ can be considered valid and appended 
            to the backwards-linked list, representing in-EVM Mina's
            commit log.
        \item In case the verification of a state proof corresponding to the
            replication packet $B_{n}$ wasn't completed sucessfully, then a
            replication packet $B_{n}$ gets rejected by the in-EVM piece of
            logic.
        \item In case there are more than a single replication packet $B_{n}$
            (e.g. $B_{n_1}$ and $B_{n_2}$) and each of them is being considered
            valid, the backward-linked list used to store such replication
            packets turns into the tree containing several branches of
            backward-linked lists ${C_1, ... , C_M}$.
        \item In case several branches ${C_1, ... , C_M}$ are introduced, the
            Mina's Ouroboros modification chain selection rule applies to pick
            the same branch the original Mina's cluster chain selection rule
            picked.
    \end{enumerate}
\end{algorithm}

$T_{n_1, n_2}$ allows to provide a successfull transaction from $\{B_{n_1}, ..., B_{n_2}\}$ to the Ethereum-based proof verificator later. 

Ouroboros' consensus protocol chain selecton rule which is supposed to handle
potentially incorrect replication packet data submitted by the user (and to keep
the in-EVM commit log consistent with the actual Mina's one) is defined as
follows:

Here, $C_{loc}$ is the local commit log sequence, $N = {C_1, ... ,C_M}$ is the list 
of potential commit log sequences to choose from. 
The function $getMinDen(C)$ outputs the minimum of all the window densities
observed thus far in $C$.

\begin{algorithm}[H]
    \caption{getMinDen(C)}
    Let $B_{last}$ be the last block in $C$.
    \begin{enumerate}
        \item if $B_{last} = G$ then // i.e., if $B_{last}$ is the genesis block 
        \item return $0$
        \item else
        \item Parse $B_{last}$ to obtain the parameter $minDen$. 
        \item return $minDen$
    \end{enumerate}
\end{algorithm}

The function $isShortRange(C,C')$ outputs whether or not the chains fork in the “short range” or not. 

\begin{algorithm}[H]
    \caption{isShortRange(C1, C2)}
    \begin{enumerate}
        \item Let $prevLockcp$ and $prevLockcp$ be the $prevLockcp$ components in the 12 last blocks of C1, C2, respectively. 
        \item if $prevLockcp = prevLockcp$ then
        \item return $T$
        \item else
        \item return $\bot$
    \end{enumerate}
\end{algorithm}

\begin{algorithm}[H]
    \caption{maxvalid-sc($C_{loc}, N = {C_1, ... , C_M}, k)$}
    \begin{enumerate}
        \item Set $C_{max} \Leftrightarrow C_{loc}$ // Compare $C_{loc}$ with each candidate chain in N
        \item for $i = 1, ... , M$ do
            if $isShortRange(C_i, C_{max})$ then // Short-range fork\\
            if $|C_i| > |C_{max}|$ then \\
            Set $C_{max} \Leftrightarrow C_{i}$ \\
            end if\\
            else //Long-range fork\\
            if $getMinDen(C_i) > getMinDen(C_{max})$ then \\
            Set $C_{max} \Leftrightarrow C_{i}$ \\
            end if \\
            end if \\
            end for \\
        \item return $C_{max}$
    \end{enumerate}
\end{algorithm}

\subsection{Purpose}

The protocol is supposed to make it possible for the users to prove a particular
transaction to the in-EVM Mina's commit log replica to be able to prove it
actually belongs to Mina's commit log.

The overview of such a mechanism is as follows:

\begin{algorithm}[H]
    \caption{Transaction Plaintext Data Proving Approach}
    \label{commitlog}
    \begin{enumerate}
        \item A user retrieves the transaction $T$ from Mina's database commit log
        \item A user compares the tranasction $T$ with the contents of the
            in-EVM Mina's commit log representation.
        \item If a trivial comparison results in a match, Mina's data from the
            transaction $T$ can be considered valid for the in-EVM usage.
        \item Otherwise, the transaction is supposed to be rejected.
    \end{enumerate}
\end{algorithm}
