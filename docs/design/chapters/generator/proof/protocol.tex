\section{RedShift Protocol}
\label{section:protocol}
\textbf{WIP}

Notations:

\begin{center}
    \begin{table}[H]
        \begin{tabular}{| l | l |}
            \hline
            $N_{\texttt{wires}}$                                                              & Number of wires (`advice columns`)                            \\
            \hline
            $N_{\texttt{perm}}$                                                               & Number of wires that are included in the permutation argument \\
            \hline
            $N_{\texttt{sel}}$                                                                & Number of selectors used in the circuit                       \\
            \hline
            $N_{\texttt{const}}$                                                              & Number of constant columns                                    \\
            \hline
            $\textbf{f}_i$                                                                    & Witness polynomials, $0 \leq i < N_{\texttt{wires}}$          \\
            \hline
            $\textbf{f}_{c_i}$                                                                & Constant-related polynomials, $0 \leq i < N_{\texttt{const}}$ \\
            \hline
            $\textbf{gate}_i$                                                                 & Gate polynomials, $0 \leq i < N_{\texttt{sel}}$               \\
            \hline
            $\sigma(\text{col : } i, \text{row : } j) = (\text{col : } i', \text{row : } j')$ & Permutation over the table \\
            \hline
        \end{tabular}
    \end{table}
\end{center}

For details on polynomial commitment scheme and polynomial evaluation scheme, we refer the reader to \cite{cryptoeprint:2019:1400}.

\paragraph{Preprocessing:}


\begin{algorithm}[h]
    \begin{enumerate}
        \item $\mathcal{L}' = (\textbf{q}_{0}, ..., \textbf{q}_{N_{\texttt{sel}}})$
        \item Let $\omega$ be a $2^k$ root of unity
        \item Let $\delta$ be a $T$ root of unity, where $T \cdot 2^S + 1 = p$ with $T$ odd and $k \leq S$
        \item Compute $N_{\texttt{perm}}$ permutation polynomials $S_{\sigma_i}(X)$ such that $S_{\sigma_i}(\omega^j) = \delta^{i'} \cdot \omega^{j'}$
        \item Compute $N_{\texttt{perm}}$ identity permutation polynomials: $S_{id_i}(X)$ such that $S_{id_i}(\omega^j) = \delta^i \cdot \omega^j$
        \item Let $H = \{\omega^0, ..., \omega^n\}$ be a cyclic subgroup of $\mathbb{F}^*$
        \item Let $Z(X) = \prod\limits{a \in H^*}(X - a)$
    \end{enumerate}
\end{algorithm}

\subsection{Prover View}

\begin{enumerate}
    \item Choose masking polynomials:
    \begin{center}
        $h_i(X) \leftarrow \mathbb{F}_{<k}[X]$ for $0 \leq i < N_{\texttt{wires}}$
    \end{center}
    \textbf{Remark}: For details on choice of $k$, we refer the reader to \cite{cryptoeprint:2019:1400}.
    \item Define new witness polynomials:
    \begin{center}
        $f_i(X) = \textbf{f}_{i}(X) + h_i(X)Z(X)$ for $0 \leq i < N_{\texttt{wires}}$
    \end{center}
    \item Add commitments to $f_i$ to $\text{transcript}$
    \item Get $\beta, \gamma \in \mathbb{F}$ from $hash(\text{transcript})$
    \item For $0 \leq i < N_{\texttt{perm}}$
    \begin{center}
        $p_i = f_i + \beta \cdot S_{id_i} + \gamma$ \\
        $q_i = f_i + \beta \cdot S_{\sigma_i} + \gamma$
    \end{center}
    \item Define:
    \begin{center}
        $p'(X) = \prod\limits_{0 \leq i < N_{\texttt{perm}}} p_i(X) \in \mathbb{F}_{<N_{\texttt{perm}} \cdot n}[X]$ \\
        $q'(X) = \prod\limits_{0 \leq i < N_{\texttt{perm}}} q_i(X) \in \mathbb{F}_{<N_{\texttt{perm}} \cdot n}[X]$
    \end{center}
    \item Compute $P(X), Q(X) \in \mathbb{F}_{<n+1}[X]$, such that:
    \begin{center}
        $P(\omega) = Q(\omega) = 1$ \\
        $P(\omega^i) = \prod\limits_{1 \leq j < i}p'(\omega^i)$ for $i \in {2, \dots, n + 1}$ \\
        $Q(\omega^i) = \prod\limits_{1 \leq j < i}q'(\omega^i)$ for $i \in {2, \dots, n + 1}$ \\
    \end{center}
    \item Compute commitments to $P$, $Q$ and add them to $\text{transcript}$.
    \item Get $\alpha_0, \dots, \alpha_5 \in \mathbb{F}$ from $hash(\text{transcript})$
    \item Get $\tau$ from $hash(\text{transcript})$
    \item Define polynomials ($F_0, \dots, F_4$ - copy-satisfability, $\texttt{gate}_0$ is $PI$-constraining gate)):
    \begin{center}
        $F_0(X) = L_1(X)(P(X) - 1)$\\
        $F_1(X) = L_1(X)(Q(X) - 1)$ \\
        $F_2(X) = P(X)p'(X) - P(X\omega)$ \\
        $F_3(X) = Q(X)q'(X) - Q(X\omega)$ \\
        $F_4(X) = L_n(X)(P(X\omega) - Q(X\omega))$ \\
        $F_5(X) = \sum\limits_{0 \leq i < N_{\texttt{sel}}} (\tau^i \cdot \textbf{q}_{i}(X) \cdot \texttt{gate}_i(X))
        + PI(X)$
    \end{center}
    \item Compute:
    \begin{center}
        $F(X) = \sum\limits_{i = 0}^5 \alpha_iF_i(X)$ \\
        $T(X) = \frac{F(X)}{Z(X)}$
    \end{center}
    \item $N_T \coloneqq \texttt{max}(N_{\texttt{perm}}, \texttt{deg}_{\texttt{gates}} - 1)$, 
		where $\texttt{deg}_{\texttt{gates}}$ is the highest degree of the degrees of gate polynomials. 
    \item Split $T(X)$ into separate polynomials $T_0(X), ..., T_{N_T - 1}(X)$\footnote{
    	Commit scheme supposes that polynomials should be degree $\leq n$}
    \item Add commitments to $T_0(X), ..., T_{N_T - 1}(X)$ to $\text{transcript}$.
    \item Get $y \in \mathbb{F}/H$ from $hash(\text{transcript})$
    \item Run evaluation scheme with the committed polynomials and $y$. \\
    \textbf{Remark}: Depending on the circuit, evaluation can be done also on $y\omega, y\omega^{-1}$.
    \item The proof is $\pi_{\texttt{comm}}$ and $\pi_{\texttt{eval}}$, where:
    \begin{itemize}
        \item $\pi_{\texttt{comm}} = \{f_{0, \texttt{comm}}, \dots, f_{N_{\texttt{wires}} - 1, \texttt{comm}},
        P_{\texttt{comm}}, Q_{\texttt{comm}}, T_{0, \texttt{comm}}, ..., T_{N_T - 1, \texttt{comm}} \}$
        \item  $\pi_{\texttt{eval}}$ is evaluation proofs for $f_0(y), \dots, f_{N_{\texttt{wires}} - 1}(y), P(y), P(y\omega), Q(y), Q(y\omega), T_0(y), \dots, T_{N_T - 1}(y)$
    \end{itemize}
\end{enumerate}

\subsection{Verifier View}

\begin{enumerate}
	\item Let $f_{0, \texttt{comm}}, \dots, f_{N_{\texttt{wires}} - 1, \texttt{comm}}$ be commitments to $f_{0}(X), \dots, f_{N_{\texttt{wires}} - 1}(X)$
    \item $\text{transcript} = \text{setup\_values} || f_{0, \texttt{comm}} || \dots || f_{N_{\texttt{wires}} - 1, \texttt{comm}}$
    \item $\beta, \gamma = hash(\text{transcript})$
    \item Let $P_{\texttt{comm}}, Q_{\texttt{comm}}$ be commitments to $P(X), Q(X)$
    \item $\text{transcript} = \text{transcript} || P_{\texttt{comm}} || Q_{\texttt{comm}}$
    \item $\alpha_0, \dots, \alpha_5 = hash(\text{transcript})$
    \item $\tau = hash(\text{transcript})$
    \item $N_T \coloneqq \texttt{max}(N_{\texttt{perm}}, \texttt{deg}_{\texttt{gates}} - 1)$, 
		where $\texttt{deg}_{\texttt{gates}}$ is the highest degree of the degrees of gate polynomials. 
    \item Let $T_{0, \texttt{comm}}, ..., T_{N_T - 1, \texttt{comm}}$ be commitments to $T_0(X), ..., T_{N_T - 1}(X)$
    \item $\text{transcript} = \text{transcript} || T_{0, \texttt{comm}} || ... || T_{N_T - 1, \texttt{comm}}$
    \item $y = hash|_{\mathbb{F}/H}(\text{transcript})$
    \item Run evaluation scheme verification with the committed polynomials and $y$ to check values
    $f_i(y), P(y), P(y\omega), Q(y), Q(y\omega), T_j(y)$.  \\
    \textbf{Remark}: Depending on the circuit, evaluation can be done also on $f_i(y\omega), f_i(y\omega^{-1})$ for some $i$.
    \item Calculate:
    \begin{center}
        $F_0(y) = L_1(y)(P(y) - 1)$ \\
        $F_1(y) = L_1(y)(Q(y) - 1)$ \\
        $p'(y) = \prod p_i(y) = \prod f_i(y) + \beta \cdot S_{id_i}(y) + \gamma$ \\
        $F_2(y) = P(y)p'(y) - P(y\omega)$ \\
        $q'(y) = \prod q_i(y) = \prod f_i(y) + \beta \cdot S_{\sigma_i}(y) + \gamma$ \\
        $F_3(y) = Q(y)q'(y) - Q(y\omega)$ \\
        $F_4(y) = L_n(y)(P(y\omega) - Q(y\omega))$ \\
        $F_5(y) = \sum\limits_{0 \leq i < N_{\texttt{sel}}} (\tau^i \cdot \textbf{q}_{i}(y) \cdot \texttt{gate}_i(y))
        + PI(y)$ \\
        $T(y) = \sum\limits_{0 \leq j < N_{T}}y^{n \cdot j}T_j(y)$
    \end{center}
    \item Check the identity:
    \begin{center}
        $\sum\limits_{i = 0}^5\alpha_iF_i(y) = Z(y)T(y)$
    \end{center}
\end{enumerate}
